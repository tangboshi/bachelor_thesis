% !TeX document-id = {11d8fae9-4ab4-415f-8574-09a217d6098d}
% !TeX TXS-program:compile = txs:///pdflatex/[--shell-escape]
\documentclass{article}

\usepackage[utf8]{inputenc}
\usepackage[T1]{fontenc}
\usepackage{color}
\usepackage{soul}
\usepackage{amsmath}
\usepackage{amssymb}
\usepackage{listings}
\usepackage{minted}
\usepackage{hyperref}
\usepackage{graphicx}
\usepackage{calc}
\usepackage{enumitem}
\usepackage{standalone}

\graphicspath{{img/}}
\setlength{\parindent}{0pt}

\begin{document}
	
\title{CW 21-25 Summary}
\author{Alexander Pastor}
\date{18.05.2017}
\maketitle
\tableofcontents
\newpage

to do:
Python: kwargs, map, zip, list comprehension
GR: hardware problem, how to exclude retransmissions from rtt, how to calculate delay if including them
GR Scenarios: think of scenarios
send Andra \& Peng the plots
git: Personal Access Tokens
latex: input
description of my work so far...

\section{Flow Chart Details}
\subsection{Counter Meanings}

In the theoretical CSMA and ALOHA files:
\begin{description}[leftmargin=!]
	\item[Counter 1:] Counts received frames.
	\item[Counter 2:] Counts good frames.
	\item[Counter 3:] Counts bad frames.
	\item[Counter 4:] Counts frames that aren't destined to this RX.
	\item[Counter 5:] Counts frames that are destined to this RX.
\end{description}

\section{Organizational}

\subsection{Device Parameters}

Remote PC IP-address: 134.130.223.135 \\
USRPs: 10.0.0.16, 10.0.0.7

\subsection{Measurement Plan}

For CW 25 the following measurement parameters were commissioned:

\begin{itemize}
	\item 2 sets of 5 measurements each with a duration of 5 minutes.
	\item SIFS: 1ms / 3ms 
	\item DIFS: 5ms  / 15ms 
	\item Backoff slot: 2s / 6ms
	\item The results should be plotted with either Matlab or Matplotlib
\end{itemize}

\textbf{Note:} The DIFS time in 802.11 can be calculated as twice the backoff time plus one SIFS time.

\section{GNU Radio Implementation Details}

\subsection{A Closer Look at PMTs}

\section{Good to Know...}

It has been a good while now, but finally I learned enough feasible stuff to write down.

\subsection{Using a static IP address for USRPs}
\begin{minted}[tabsize=4]{bash}
#Get to know your interface names
ifconfig 
#E.g. set eth0 interface IP to 10.0.0.100/24
#Where up opens the eth0 interface 
#and is not necessary if it showed up when using infconfig
ifconfig eth0 10.0.0.100 netmask 255.255.255.0 up

###Just FYI:
#Shutdown network interface
ifconfig eth0 down
##List all interfaces
ifconfig -a
#OR
ip link show
\end{minted}

\subsection{Efficient Remote Control}

\begin{minted}[tabsize=4]{bash}
##Opening a remote connection with X forwarding
#Using -X instead of -Y adds security check, but reduces performance
#Use -C to get it more stable (gzip-compression)
#Optional -c aes128-ctr: for AES 128 encryption
#Optional -4: forces the usage of IPv4 addresses
ssh -YC4c aes-128ctr inets@134.130.223.135

##In a second terminal mount file system to local folder
#This has the advantage of the ability to treat files as though they were local
mkdir -p mnt/134.130.223.135
cd mnt/134.130.223.135

#Tip: in the next line for the last argument type . and then expand with Tab :)
#If it is necessary sshfs also allows user-mapping if file ownership is an issue
sshfs inets@134.130.223.135:/home/inets /home/alex/mnt/134.130.223.135/

#Then edit all you like as though the mounted partition was local
cd source/gr-inets/lib
atom *some_file*

\end{minted}

To further avoid any repetitive commands I added to my \href{http://alexander-pastor.de/convenient-linux-game-launcher/}{launcher tool}:

\begin{minted}[tabsize=4]{bash}
"inets")
	if ( mount | grep inets  )
		then
		echo "The mount point is in use, confirm unmount with your password."
		sudo umount /home/alex/mnt/134.130.223.135
	fi
	echo "Please enter the server password to mount the target directory."
	sshfs inets@134.130.223.135:/home/inets /home/alex/mnt/134.130.223.135
	gnome-terminal --tab -e "bash -c 'ssh -YC4 inets@134.130.223.135'" 
	--tab --working-directory=/home/alex/mnt/134.130.223.135
	sleep 1
	exit
;;
\end{minted}

\subsection{Listing Directories Only}

\href{https://stackoverflow.com/questions/14352290/listing-only-directories-using-ls-in-bash-an-examination}{Stackoverflow Source}

\begin{minted}[tabsize=4]{bash}
##Possibility 1 (fastest):
echo */
#List all subsubfolders as well:
echo */*/

##Possibility 2 (straightforward ls):
ls -d */

##Possibility 3
#where ^ means beginning of a line
ls -l | grep "^d"

##Possibility 4
#If you need to list and process all directories in a bash-script (slow)
for i in $(ls -d */); do echo ${i%%/}; done
\end{minted}

\subsection{Redirecting GRC Stuff to Files}
If one wants to do heavy analysis, built-in Linux console tools including, but not limited to utilities such as \verb|awk|, \verb|grep|, \verb|sed|, \verb|cat| might be a great help.

Since executing a GRC flowgraph is running a generated python script, we can achieve our goal easily by the simple means of:

\begin{minted}[tabsize=4]{bash}
## Just as a concept
#These could be some lines in my alohaTestSuite.sh
mkdir -p logs
time python2 theoretical_aloha_rx.py &> logs/theoretical_aloha_rx.log
time python2 theoretical_aloha.py &> logs/theoretical_aloha.log
\end{minted}

\subsection{Atom: Folder-Wide Search for Substrings}
Use the Ctrl+Shift+F shortcut and all "project folders" will be searched. To add multiple folders use the Ctrl+Shift+A shortcut.

\end{document}