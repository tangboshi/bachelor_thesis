% !TeX TXS-program:compile = txs:///pdflatex/[--shell-escape]
\documentclass{article}

\usepackage[utf8]{inputenc}
\usepackage[T1]{fontenc}
\usepackage{color}
\usepackage{soul}
\usepackage{amsmath}
\usepackage{amssymb}
\usepackage{listings}
\usepackage{minted}
\usepackage{hyperref}
\usepackage{graphicx}
\usepackage{calc}
\usepackage{enumitem}
\usepackage{standalone}
\usepackage{outlines}
\usepackage{enumitem}

\setenumerate[1]{label=\arabic*}
\setenumerate[2]{label*=.\arabic*}
\setenumerate[3]{label*=.\arabic*}
\setenumerate[4]{label*=.\arabic*}

\graphicspath{{img/}}
\setlength{\parindent}{0pt}

\begin{document}
	
\title{Thesis Draft}
\author{Alexander Pastor}
\maketitle

\clearpage
\tableofcontents
\clearpage

\section{Abstract}

%% Real World Motivation
% Since when is the "lease" of frequency spectrum a must?
The demand for higher data transfer rates is ever growing. The wireless segment is no exception - in fact it is a rapidly increasing market. Most frequency bands are licensed for exclusive use. While this practice has the upside of ensuring interference free conduct for the licensed use, it at the same time may prevent innovative concurrent technologies evolving. Furthermore, the licensed channel may be operating under unsaturated traffic in both - the spatial and temporal - domains. This gives rise to the following question: Should other users be allowed to occupy a licensed band whenever the license holder is not making full use of the licensed channel's capacity?

%% Example LTE-U and LAA
% 

%% MAC-Protocols
In a nutshell, all boils down to the question how different wireless technologies may coexist based on their medium access protocols.

\section{Introduction}

To further motivate the let me present you a few introductory questions and their respective answers.

\subsection{Why Wireless?}

\section{Background}

\subsection{MAC Layer in the OSI-Stack}


\end{document}