% !TeX TXS-program:compile = txs:///pdflatex/[--shell-escape]
\documentclass{article}

\usepackage[utf8]{inputenc}
\usepackage[T1]{fontenc}
\usepackage{color}
\usepackage{soul}
\usepackage{amsmath}
\usepackage{amssymb}
\usepackage{listings}
\usepackage{minted}
\usepackage{hyperref}
\usepackage{graphicx}
\usepackage{calc}
\usepackage{enumitem}
\usepackage{standalone}
\usepackage{outlines}
\usepackage{enumitem}

\setenumerate[1]{label=\arabic*}
\setenumerate[2]{label*=.\arabic*}
\setenumerate[3]{label*=.\arabic*}
\setenumerate[4]{label*=.\arabic*}

\graphicspath{{img/}}
\setlength{\parindent}{0pt}

\begin{document}
	
\title{Thesis Draft}
\author{Alexander Pastor}
\maketitle

\clearpage
\tableofcontents
\clearpage

\section{Abstract}

%% Real World Motivation
% Since when is the "lease" of frequency spectrum a must?
The demand for higher data transfer rates is ever growing. The wireless segment is no exception - in fact it is a rapidly increasing market. Most frequency bands are licensed for exclusive use. While this practice has the upside of ensuring interference free conduct for the licensed use, it at the same time may prevent superior technologies from evolving and gaining acceptance, simply by preventing those technologies to be used on that band. Furthermore, the licensed channel may be operating under spatially and temporally unsaturated traffic and thus wasting capacity that may be used otherwise. 

\medskip

This gives rise to the following question: Should other users be allowed to occupy a licensed band whenever the license holder is not making full use of the licensed channel's capacity? On the same page, should a license holder make use of "unlicensed capacities" if that fortifies his quality of service, whilst claiming not harming everyone else?

\medskip

%% Example LTE-U and LAA
A proof for the relevance of this question is the recent "LTE Unlicensed" debate. LTE Unlicensed equipment vendors such as Qualcomm claim that LTEU leads to higher spectral efficiency. Additionally, LTEU would be a better neighbor to WiFi than WiFi itself in terms of joint and even individual throughput of both LTEU and WiFi. Google, as well as the WiFi community dismiss these claims and issued their own study, which concludes that LTEU would "kill" WiFi's performance should LTEU operate in the 5 GHz band. Both parties accuse each other of having chosen unrealistic parameters in their respective studies.

\medskip

%% MAC-Protocols

In a nutshell, all boils down to the question how different wireless technologies may coexist based on their medium access protocols. In this thesis we take a close look at a CSMA/CA type of protocol as used in 802.11.

\clearpage

\section{Introduction}

To further motivate the let me present you a few introductory questions and their respective answers.

\subsection{Why Wireless?}

\section{Background}

\subsection{MAC Layer in the OSI-Stack}


\end{document}