The demand for higher wireless transmission rates and capacity is growing rapidly. As a consequence, options to more efficiently make use of the limited frequency resources are being explored. Recently, it was proposed that operators of cellular networks should make use of license-free frequency bands that were not originally designated for their purposes. However, due to the exemption of licenses these bands are densely populated and measures for peaceful coexistence with incumbent technologies must be taken. In the case of the license-free 2.4 GHz band multiple technologies already coexist, namely IEEE 802.11 (Wi-Fi), Bluetooth and IEEE 802.15.4 (ZigBee). With this in mind, it is only natural to take mechanisms of technologies into account that were designed for contention-based coexistence, in our case the CSMA/CA protocol used in Wi-Fi devices. 
This thesis experimentally examines how different medium access control protocols determine the overall performance of the nodes. To this end, we use USRP software-defined radio devices and GNU Radio to employ different medium access control protocols, namely CSMA/CA, 1-persistent CSMA and ALOHA in various combinations on two links, where the focus lies on the influence of CSMA timing aspects. 
Our measurements reveal that the appropriate choice of timing parameters is crucial to the performance of the devices in different traffic situations concerning throughput and frame delays. On the one hand, it is important that the transmission channel is not idle due to excessive sensing periods in spite of backlogged nodes. On the other hand, it is important to prevent nodes starting transmission prematurely causing collisions due to inaccuracies related to the time granularity of the overall system.  