The demand for higher wireless transmission rates and capacities is growing rapidly. As a consequence, options to more efficiently make use of the limited frequency resources are being explored. One possibility for operators is to make use of license-free frequency bands that were not originally designated for their purposes. However, due to the exemption of license fees these bands are densely populated and measures for peaceful coexistence with "indigenous" technologies must be taken. With this in mind, it is only natural to take mechanisms of technologies into account that were designed for contention-based coexistence, in our case the CSMA/CA protocol used in IEEE 802.11 (WLAN) devices. 
This thesis experimentally examines how different medium access control (MAC) protocols and their mechanisms determine the overall performance of the nodes. To this end, we use software-defined radio (SDR) devices and software (GNU Radio) to employ different MAC protocols (CSMA/CA, 1-persistent CSMA and ALOHA) on the devices, where the focus lies on the influence of timing aspects. Our setup consists of two links on which we employ the MAC protocols in various combinations.
Our measurements reveal that the appropriate choice of timing parameters is crucial to the performance of the devices in different traffic situations concerning individual and aggregate throughput and frame delays. On the one hand, it is important that the transmission channel is not idle in spite of backlogged nodes due to excessive waiting periods. On the other hand, it is important to prevent nodes starting transmission prematurely causing collisions due to inaccuracies related to the time granularity of the overall system.  