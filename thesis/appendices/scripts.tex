\chapter{Bash and Python Scripts}

This appendix aims at giving insight in selected scripts used in the three phases of the measuring, data processing and plotting process. The basic principle, however, is depicted in section \ref{sec:script-system}. Minor edits were made for format and aesthetic reasons.

\begin{lstlisting}[language=Bash,caption=measure.sh]
echo "remote_measurement is set to "$remote_measurement"."

function setup_remote_connection
{
  reset
  sshpass -p "inets" ssh -$remote_flags $remote_user@$remote_ip "bash -s" < remote_measurement_$link.sh
}

function prepare_measurement
{
    reset
    measurement_counter=0
    ## let's make sure all the directories exist
    printf "\nchecking if paths exists...\n"

    #let's first make absolutely sure the raw data source path exists
    if [ ! -d $raw_data_source_path ];
      then
        mkdir -p $raw_data_source_path
        echo $raw_data_source_path" created."
      else
        rm -r $raw_data_source_path/*
    fi

    if [ -d $plot_directory_path ];
      then
        echo $plot_directory_path" already existed!"
        cd $plot_directory_path
        # create measurement directory
        while [ -d $measurement_counter ]; do
            measurement_counter=$(($measurement_counter+1))
        done
        export measurement_counter;
    fi

    if [ -d $log_path ];
      then
        echo $log_path" already existed!"
      else
        mkdir -p $log_path
        echo $log_path" directory created."
    fi

    mkdir -p $plot_directory_path/$measurement_counter
    echo $plot_directory_path/$measurement_counter" directory created."

    mkdir -p $data_source_path/$measurement_counter
    echo $data_source_path/$measurement_counter" directory created."

    mkdir -p $jobs_open_path
    mkdir -p $jobs_done_path

    ## let's check if measurement script is defined
    # if $measurement_scripts undefined:
    # go through directory and list all python files
    if [ -z ${measurement_scripts+x} ];
      then
        echo  "no measurement scripts set,
              going through files inside of $locate_base_path."
        echo "please add a the full path of one of the files to \$scritps."
        locate -r "$locate_base_path" | grep "\.py$"
        echo "terminated. ding dong"
        exit -1
    fi

    printf "\n"
}

function measure
{
  local prematurely_aborted=0

  for ((x = 1 ; x <= $measurement_repetitions ; x += 1)); do

    # get pid to later kill it
    for i in "${measurement_scripts[@]}"
    do
      python $measurement_script_path/$i &
    done

    for ((y = $timer ; y > 0 ; y -= 1)); do
      echo "measurement $x/$measurement_repetitions complete in $y second(s)."
      if [ $check_if_prematurely_aborted -eq 1 ];
        then
          if $(ps -p ${measurement_scripts_pid[*]}) | grep ${measurement_scripts_pid[*]};
            then
              :
            else
              prematurely_aborted=1
              echo  "Scripts were killed prematurely. Measurement may be incomplete."
              break
          fi
      fi
      sleep 1
    done

    kill $(jobs -p)

    # save this measurement's data to special folder
    mkdir -p $data_source_path/$measurement_counter/$x
    echo  "measurement $x raw data directory created $data_source_path/$measurement_counter/$x/."

    echo $raw_data_source_path
    echo $(ls $raw_data_source_path | egrep "*_$link.txt")

    cd $raw_data_source_path
    mv -v $(ls | egrep "*_$link.txt") $data_source_path/$measurement_counter/$x/
    cp -v $(ls | egrep "sniffer") $data_source_path/$measurement_counter/$x/
    if [ "$receiver_mode" == "single" ];
      then
        cp -v $(ls | egrep "receiver") $data_source_path/$measurement_counter/$x/
    fi
    echo  "measurement $x raw data moved to $data_source_path/$measurement_counter/$x/."
    printf "\n"
    if [ $prematurely_aborted -eq 1 ];
      then
        if [ $plot_if_prematurely_aborted -eq 0 ];
          then
            echo "plotting if measurement prematurely aborted set to false."
            echo "terminated."
            exit -1
        fi
    fi

  done

  #exit remote connection
  if [ $remote_measurement -eq 1 ]; then
    echo "remote_measurement is set to "$remote_measurement"."
    exit
  fi
}

function plot
{
  ## plot the results
  echo "now processing results..."

  # call the plotting scripts as data
  #echo "starting to generate plots..."
  echo "plotting python should be: "$plot_py" ("$os")."

  for i in ${plot_scripts[@]}; do
    bash -c "$plot_py $plot_py_path/$i"
  done

  echo "+------------------+"
  echo "|plotting completed|"
  echo "+------------------+"
}

function cleanup
{
    ##cleaning up the mess you created!
    #kill all child proceesses
    echo "staring cleanup..."
    echo "killing all lingering child processes..."
    killall -9 -g $0
    cd $this_path
    exit
}
trap cleanup sighup sigint sigkill;
trap "cd $this_path" exit;

function main
{
  # clear up console
  #reset
  # check if jobs_open directory is empty
  if [ ! "$(ls -a $jobs_open_path)" ]; then
    echo "there seem to be no open jobs. measuring with default parameters."
    prepare_measurement
    #take measurements
    measure | tee -a $log_path/default_$measurement_counter.log
    # create plot if desired
    if [ $plot_enabled -eq 1 ]; then
      plot | tee -a $log_path/default_$measurement_counter.log; fi
  else
    prepare_measurement
    echo "open jobs detected! let's get to work..."
    jobs=$jobs_open_path/*
    for job in $jobs; do
      source $job;
      job_name=$(echo $job | rev | cut -d"/" -f1 | rev )
      log=$log_path/$job_name"_"$measurement_counter.log
      #echo $job_name
      cat $job | tee -a $log
      cat measurement_$link.conf | tee -a $log
      measure | tee -a $log
      if [ $plot_enabled -eq 1 ]; then
        plot | tee -a $log
      fi
      if [ $move_after_job_done -eq 1 ]; then
        cp $job $plot_directory_path/$measurement_counter/
        mv $job $jobs_done_path/
      fi
      export measurement_counter=$((measurement_counter++))
    done
  fi
}

if [ $debug_mode -eq 1 ]; then
  echo "+-----------------+"
  echo "|debug mode active|"
  echo "+-----------------+"
fi

if [ $remote_measurement -eq 1 ]; then
  # call to main included here
    setup_remote_connection
  else
    main
fi

\end{lstlisting}

\begin{lstlisting}[language=Python,caption=evaluation.py]
import numpy as np
import myplot
import os

import rtt
import throughput as tp
import channel_occupation
import backoff
import sniffer

# From Bash
measurement = [int(os.environ["measurement_counter"])]
links = [int(os.environ["link"])]
repetitions = int(os.environ["measurement_repetitions"])
data_source_path = os.environ["data_source_path"]
plot_path = os.environ["plot_directory_path"]+"/"+os.environ["measurement_counter"]+"/"
plot_type = ["cdf", "boxplot"]
throughput_data_files = os.environ["throughput_data_files"].split(",")
rtt_data_files = os.environ["rtt_data_files"].split(",")
co_data_files = os.environ["co_data_files"].split(",")
sniffer_data_files = os.environ["sniffer_data_files"].split(",")
retxs_data_files = os.environ["retxs_data_files"].split(",")
show_plot = int(os.environ["show_plot_after_measurement"])
rtt_mode = os.environ["rtt_mode"]
max_retxs = 6
eval_mode = "live"
timer = int(os.environ["timer"])
receiver_mode = os.environ["receiver_mode"]

#From Python
plot_pdf = False
boxplot_xticks = [ "measurement "+str(index) for index in measurement ]
legend_labels = [ tick.replace("\n", ", ") for tick in boxplot_xticks ]

custom_legend_coordinates = {
    "rtt":                 [0.24,0.85,"upper left"],
    "packet_loss":         [1,0,"lower right"],
    "retxs":               [1,0,"lower right"],
    "throughput":          [1,0,"lower right"],
    "diagnosis_sender":    [1,0,"lower right"],
    "diagnosis_receiver":  [1,0,"lower right"],
    "backoff":             [1,0,"lower right"],
    "channel_occupation":  [1,0,"lower right"],
    "sniffer":             [1,0,"lower right"]
}

create_plots = {
    "rtt":                  False,
    "packet_loss":          False,
    "retxs":                False,
    "throughput":           True,
    "diagnostic":           True,
    "backoff":              True,
    "channel_occupation":   True,
    "sniffer":              True
}

channel_occupation_mode = {
    "occupation_mode":  ["overview", "zoom"],
    "zoom":             [5,7],
    "zoom_mode":        "interval",
    "zoom_interval":    2
}

sniffer_settings =  {
    "sniffer_mode":             ["physical", "smoothed"],
    "link":                     1,
    "zoom":                     [0.0,timer*repetitions],
    "zoom_mode":                "interval",
    "zoom_interval":            2,
    "smoothing_difference":     0.0001,
    "smoothing_derivative":     0.01,
    "smoothing_range":          [0.0010,0.0013]
}

#Unimplemented, use later
annotations_below   = []
annotations_other   = []

eval_dict = {
    "measurement":              measurement,
    "repetitions":              repetitions,
    "data_source_path":         data_source_path,
    "xticks":                   boxplot_xticks,
    "legend":                   legend_labels,
    "annotations_below":        annotations_below,
    "annotations_other":        annotations_other,
    "throughput_data_files":    throughput_data_files,
    "retxs_data_files":         retxs_data_files,
    "rtt_data_files":           rtt_data_files,
    "show_plot":                show_plot,
    "legend_coordinates":       custom_legend_coordinates,
    "create_plots":             create_plots,
    "links":                    links,
    "rtt_mode":                 rtt_mode,
    "channel_occupation_mode":  channel_occupation_mode,
    "co_data_files":            co_data_files,
    "sniffer_data_files":       sniffer_data_files,
    "sniffer_settings":         sniffer_settings,
    "timer":                    timer,
    "plot_pdf":                 plot_pdf
}

for index,a_plot_type in enumerate(plot_type):
    if plot_type[index] == "cdf":
        grid                = True
    else:
        grid                = True

    eval_dict["plot_type"]  = [plot_type[index]]
    eval_dict["plot_path"]  = plot_path
    eval_dict["grid"]       = grid

    if create_plots["backoff"] == True:
        print("Creating backoff plot!")
        backoff.backoff(**eval_dict).plot()
    if create_plots["rtt"] == True:
        print("Creating rtt plot!")
        rtt.rtt(**eval_dict).plot()
    if create_plots["throughput"] == True:
        print("Creating throughput plot!")
        tp.tp(**eval_dict).plot()

# The plots with only one plot type!
if create_plots["channel_occupation"] == True:
    print("Creating channel occupation plot!")
    channel_occupation.channel_occupation(**eval_dict)
if create_plots["sniffer"] == True:
    print("Creating sniffer energy plot!")
    sniffer.sniffer(**eval_dict)

print("Done.")
\end{lstlisting}

\begin{lstlisting}[language=Python,caption=sniffer.py]
import numpy as np
import subprocess
import myplot
import os.path as pt
import lines

print("Hello from throughput.py!")

class tp:
    def __init__(self,**kwargs):
        # variables left out for brevity's sake.
        # pdb.set_trace()

    def calc(self):
        print("I'll create a numpy array!")
        self.data = np.zeros(shape=(len(self.measurement),self.repetitions))
        self.sender_diagnosis = np.zeros(shape=(len(self.measurement),self.repetitions))
        self.receiver_diagnosis = np.zeros(shape=(len(self.measurement),self.repetitions))
        print("Done.")

        if self.receiver_mode == "single":
            link = ""
        else:
            link = "_"+str(self.links[index])

        for index,single_measurement in enumerate(self.measurement):
            for i in range(self.repetitions):
                print("I'll open files from iteration "+str(i+1)+" for you!")
                print("file_path:")
                file_path = self.data_source_path+'/'+str(self.measurement[index])+'/'+str(i+1)+'/'
                data_sent_file_path = file_path+self.throughput_data_files[0]+link+".txt"
                ack_received_file_path = file_path+self.throughput_data_files[1]+"_"+str(self.links[index])+".txt"
                print(data_sent_file_path)
                print(ack_received_file_path)

                data_received_file_path = file_path+self.diagnosis_files[0]+link+".txt"
                ack_sent_file_path = file_path+self.diagnosis_files[1]+link+".txt"

                if pt.isfile(ack_received_file_path):
                    print("Let's count some ACKS!")
                    ackcount = lines.linecount(ack_received_file_path)
                else:
                    # no acks found
                    print("ACK file not found at "+ack_received_file_path+".")
                    ackcount = 0
                    self.data[index,i] = 0
                if pt.isfile(data_sent_file_path):
                    print("Let's count some data!")
                    datacount = lines.linecount(data_sent_file_path)
                    print("ackcount: "+str(ackcount))
                    print("datacount: "+str(datacount))
                    # *8 => bytes --> bits ||| /1000 => 1 --> kilo
                    self.data[index,i] = min(datacount,ackcount)*self.packet_size/self.timer*8/1000
                else:
                    # no data sent off
                    print("Data file not found at "+data_sent_file_path+".")
                    datacount = 0
                    self.data[index,i] = 0

                if not self.receiver_mode == "single":
                    if pt.isfile(data_received_file_path) and pt.isfile(ack_sent_file_path) and datacount > 0 and ackcount > 0:
                        print("Hoorray, diagnosis files found!")
                        receiver_datacount = lines.linecount(data_received_file_path)
                        receiver_ackcount = lines.linecount(ack_sent_file_path)
                        self.sender_diagnosis[index,i] = 100 - (receiver_ackcount - ackcount)/receiver_ackcount*100
                        self.receiver_diagnosis[index,i] = 100 - (datacount - receiver_datacount)/datacount*100
                    else:
                        print("Diagnosis files incomplete or missing :( \
                        (or no sender-side received acks/ receiver-side received data)!")
                else:
                    print("Diagnosis based on receiver is not possible in a single receiver, multiple sender scenario.")


        print(self.data)

    def plot(self):
        self.calc()
        all_data = []

        for val in self.data:
            all_data.extend(val)

        #print("all_data:")
        #print(all_data)

        print("Let's plot! ;-)")
        if self.create_plots == True or self.create_plots["throughput"] == True:
            myplot.myplot(  data=self.data,
                plottype=self.plot_type,
                title="Throughput",
                xlabel="throughput [kbit/s]",
                ylabel="throughput [kbit/s]",
                savepath=self.plot_path+"/",
                show=self.show_plot,
                grid=self.grid,
                xticks=self.xticks,
                legend=self.legend,
                legend_loc=self.legend_loc,
                annotations_below=self.annotations_below,
                annotations_other=self.annotations_other,
                legend_coordinates=self.legend_coordinates["throughput"],
                eval_mode=self.eval_mode,
                xlims=[0,135])

        if not self.receiver_mode == "single":
            if self.create_plots == True or self.create_plots["diagnostic"] == True:
                myplot.myplot(  data=self.receiver_diagnosis,
                    plottype=self.plot_type,
                    title="Receiver Receiving Score (Inverse = Data Loss)",
                    xlabel="score [%]",
                    ylabel="score [%]",
                    savepath=self.plot_path+"/",
                    show=self.show_plot,
                    grid=self.grid,
                    xticks=self.xticks,
                    legend=self.legend,
                    legend_loc=self.legend_loc,
                    annotations_below=self.annotations_below,
                    annotations_other=self.annotations_other,
                    legend_coordinates=self.legend_coordinates["diagnosis_receiver"],
                    eval_mode=self.eval_mode)

            myplot.myplot(  data=self.sender_diagnosis,
                plottype=self.plot_type,
                title="Sender Receiving Score (Inverse = ACK Loss)",
                xlabel="score [%]",
                ylabel="score [%]",
                savepath=self.plot_path+"/",
                show=self.show_plot,
                grid=self.grid,
                xticks=self.xticks,
                legend=self.legend,
                legend_loc=self.legend_loc,
                annotations_below=self.annotations_below,
                annotations_other=self.annotations_other,
                legend_coordinates=self.legend_coordinates["diagnosis_sender"],
                eval_mode=self.eval_mode)
\end{lstlisting}

\begin{lstlisting}[language=Python,caption=myplot.py]
import matplotlib.patches as mpatches
import matplotlib.pyplot as plt
import numpy as np
import os
import math

class myplot:
    def __init__(   self, plottype, data, bins="",
                    title="", xlabel="", ylabel="",
                    show=False, savepath=False, data_x=None, **kwargs
                ):

        print("Hello from myplot.py!")

        self.data               = np.asarray(data).transpose()
        self.data_x             = data_x

        print("Title is '"+title+"'.")

        if  (title == "Retransmissions per Frame"
            or len(data) == 1
            or "bar"  in plottype
            or "hist" in plottype
            or "broken_barh" in plottype):
            print("Keeping original data format instead of transposing.")
            self.data               = data

        self.bins               = bins
        self.plottype           = plottype
        self.xlabel             = xlabel
        self.ylabel             = ylabel
        # Let's have reasonable figure dimensions
        self.fig, self.ax       = plt.subplots(figsize=(9,6))
        self.kwargs             = kwargs
        self.grid               = kwargs.get("grid", False)
        self.legend             = kwargs.get("legend", [])
        self.xticks             = kwargs.get("xticks", [])
        self.legend_loc         = kwargs.get("legend_loc", "best")
        self.annotations_below  = kwargs.get("annotations_below", [])
        self.annotations_other  = kwargs.get("annotations_other", [])
        self.legend_coordinates = kwargs.get("legend_coordinates", False)
        self.timer              = kwargs.get("timer", 300)
        self.repetitions        = kwargs.get("repetitions", 5)
        self.eval_mode          = kwargs.get("eval_mode", "belated")
        self.xlims              = kwargs.get("xlims", False)
        self.savepath           = savepath,
        self.plot_pdf           = kwargs.get("plot_pdf", False)

        plottypes = {
            "hist":         lambda: self.hist(),
            "line":         lambda: self.line(),
            "cdf":          lambda: self.cdf(),
            "pdf":          lambda: self.pdf(),
            "boxplot":      lambda: self.boxplot(),
            "debug":        lambda: self.debug(),
            "bar":          lambda: self.bar(),
            "broken_barh":  lambda: self.broken_barh(),
            "line_xy":      lambda: self.line_xy()
        }

        titles = {
            "cdf":          "CDF",
            "line":         "Line Chart",
            "line_xy":      "Line Chart",
            "hist":         "Histogram",
            "pdf":          "PDF",
            "boxplot":      "Boxplot",
            "debug":        "Debug",
            "bar":          "Bar Chart",
            "broken_barh":  "Gantt Chart"
        }

        for aplot in plottype:
            #print("single plot in plottype array is:"+str(aplot))
            self.title = title+" "+titles[aplot]
            self.savename = self.title
            self.title = ""
            plottypes[aplot]()
            #set axis limits
            self.ax.set_ylim(ymin=0)
            if self.xlims != False and not (aplot in ["boxplot", "bar"]):
                self.ax.set_xlim(self.xlims[0], self.xlims[1])
            else:
                self.ax.set_xlim(xmin=0)
            plt.tight_layout()
            self.ax.xaxis.grid(self.grid, linestyle="dashdot")
            self.ax.yaxis.grid(self.grid, linestyle="dashdot")
            if not aplot == "boxplot" and not aplot == "broken_barh":
                if len(np.asarray(self.data).transpose()) == len(self.legend):
                    if self.legend_coordinates == False:
                        box = self.ax.get_position()
                        self.ax.set_position([
                            box.x0,
                            box.y0+box.height*0.3,
                            box.width,
                            box.height*0.7
                        ])
                        self.ax.legend(fancybox=True,
                                    loc='upper center',
                                    bbox_to_anchor=(0.5, -0.15))
                    else:
                        if self.legend_coordinates[2] != "best":
                            self.ax.legend(fancybox=True,
                                        loc=self.legend_coordinates[2],
                                        bbox_to_anchor=(self.legend_coordinates[0],
                                                        self.legend_coordinates[1]))
                        else:
                            self.ax.legend(fancybox=True,loc="best")
                else:
                    print ( "len(self.data) = "
                            + str(len(np.asarray(self.data).transpose()))
                            + " and len(self.legend) = "
                            + str(len(self.legend))
                            +" don't match!")
            # Add anotations:
			self.annotate(annotation)
            # Save and show plot
            if(savepath):
                print("***savepath***")
                print(savepath)
                self.save(savepath, aplot)
            if(show):
                self.show()

            #plt.close(self.fig)
            self.fig.clear()
            #print(self.data)

    def bar(self):

        colors = ['steelblue', 'orange', 'green', 'red', 'purple', 'brown', 'pink']
        color_repetitions = math.ceil(len(self.data)/len(colors))
        colors = color_repetitions * colors

        print(self.data)
        for idx,val in enumerate(self.data):
            data_points=len(self.data[idx])
            width=5/data_points
            index=np.arange( data_points )

            self.ax.bar(left=idx*width+width/2,
                height=val,
                color=colors[idx],
                alpha=0.5
             )

            self.setLabels(ylabel=self.ylabel,
                xlabel="measurement",
                title=self.title
            )

    def broken_barh(self):
        plot_data = []
        debug_data = []
        for index,item in enumerate(self.data["occupation_starting"]):
            plot_data.append(list(zip(self.data["occupation_starting"][index], self.data["occupation_durations"][index])))

        for index,item in enumerate(self.data["acks_received"]):
            debug_data.append(list(zip(self.data["acks_received"][index], self.data["acks_received_bar_width"][index])))

        print("plot_data:")
        #print(plot_data)
        print("debug_data:")
        #print(debug_data)
        print("data_len:")
        data_len = len(plot_data)
        print(data_len)
        debug_len = len(debug_data)
        print("debug_len:")
        print(debug_len)

        print("data and ack lengths:")
        for index,item in enumerate(plot_data):
            if index % 2 == 0:
                print("data index "+str(index)+":")
            else:
                print("ack index "+str(index)+":")
            print(len(item))

        for index, item in enumerate(debug_data):
            print ("debug data index "+str(index)+":")
            print(len(item))
            #print(item)

        self.ax.set_ylim(10, 5*data_len+20)
        if self.xlims == False:
            self.ax.set_xlim(0, self.timer*self.repetitions)
        self.ax.xaxis.grid(self.grid, linestyle="dashdot")
        self.ax.yaxis.grid(self.grid, linestyle="dashdot")

        self.ax.set_yticks([x*10+15 for x in range(int(data_len/2))])
        self.xticks = [tick.replace(",", ",\n") for tick in self.xticks]
        self.ax.set_yticklabels([self.xticks[index] for index in range(int(data_len/2))])

        self.setLabels(
            xlabel="time[s]",
            title=self.title
        )

        colors = ['blue','red','black']
        transparency=0.7

        patch_a  = mpatches.Patch(color=colors[0], alpha=transparency, label="data sent")
        patch_b  = mpatches.Patch(color=colors[1], alpha=transparency, label='ack sent')
        patch_c  = mpatches.Patch(color=colors[2], alpha=transparency, label="ack received")

        if self.legend_coordinates[2] != "best":
            self.ax.legend( handles=[patch_a,patch_b,patch_c],
                            fancybox=True,
                            loc=self.legend_coordinates[2],
                            bbox_to_anchor=(self.legend_coordinates[0],
                                            self.legend_coordinates[1]))
        else:
            self.ax.legend( handles=[patch_a,patch_b,patch_c],
                            fancybox=True,
                            loc="best")

        for index,item in enumerate(plot_data):
            print("Added (data) set with index "+str(index)+" to plot.")
            if index % 2 == 0:
                self.ax.broken_barh(item,((index+1)*5+5,13), facecolors=colors[0], alpha=0.5)
            elif (index-1) % 2 == 0: # well else should be enough here :)
                self.ax.broken_barh(item,((index)*5+5,13), facecolors=colors[1], alpha=0.5)

        for index,item in enumerate(debug_data):
            print("Added (debug) set with index "+str(index)+" to plot.")
            if index % 2 == 0:
                self.ax.broken_barh(item,((index+1)*5+5,13), facecolors=colors[2], alpha=0.5)
            elif (index-1) % 2 == 0:
                self.ax.broken_barh(item,((index)*5+5,13), facecolors=colors[2], alpha=0.5)

        plt.tight_layout()

    def line(self):
        from scipy.interpolate import interp1d
        x = np.arange(1, len(self.data)+1, 1)
        y = self.data
        f = interp1d(x,y)
        plt.plot(x, f(x), 'k')

        self.setLabels( xlabel="measurement",
                        ylabel=self.ylabel,
                        title=self.title
        )

    def line_xy(self):
        from scipy.interpolate import interp1d
        x = self.data_x
        y = self.data
        f = interp1d(x,y)
        plt.plot(x, f(x), 'k')
        self.setLabels( xlabel=self.xlabel,
                        ylabel=self.ylabel,
                        title=self.title
        )

    def hist(self):
        self.n, self.bins, self.patches = self.ax.hist(x=self.data,
                        bins=self.bins,
                        normed=1,
                        histtype='step',
                        cumulative=True,
                        label=self.legend)
        self.setLabels( xlabel=self.xlabel,
                        ylabel="cumulative density",
                        title=self.title)
        print(self.patches)

    def cdf(self):
        #print(self.data)
        print(self.legend)
        markers = ["x","v","o","^","8","s","p","+","D","*"]
        linestyles = ["-", "--", "-.", ":","-", "--", "-.", ":","-", "--"]
        linewidths = [1.8,1.65,1.5,1.35,1.2,1.05,1,0.9,0.8,0.75]

        if self.eval_mode == "belated":
            cdf_data = np.asarray(self.data).transpose()
        if self.eval_mode == "live":
            cdf_data = np.asarray(self.data).transpose()
            if self.title == "Retransmissions per Frame":
                cdf_data = [cdf_data]
        if len(self.data) == 1:
            print("Hooray, my title is "+self.title+".")
            cdf_data = self.data

        if len(cdf_data) > 1:
            for index,item in enumerate(cdf_data):
                print("index:"+str(index))
                print("___markers____")
                print(markers[index])
                x = np.sort(item)
                y = np.arange(1,len(x)+1) / len(x)
                x = np.insert(x,0,x[0])
                y = np.insert(y,0,0)
                self.plot = plt.step(x,
                    y,
                    marker=markers[index],
                    linestyle=linestyles[index],
                    linewidth=linewidths[index],
                    markevery=range(1,len(x)),
                    label=self.legend[index])
        else:
            x = np.sort(cdf_data[0])
            y = np.arange(1,len(x)+1) / len(x)
            x = np.insert(x,0,x[0])
            y = np.insert(y,0,0)
            print(x)
            print(y)
            self.plot = plt.step(x,
                    y,
                    marker=markers[0],
                    linestyle=linestyles[0],
                    linewidth=linewidths[0],
                    markevery=range(1,len(x)),
                    label=self.legend[0])

        self.setLabels( xlabel=self.xlabel,
                        ylabel="cumulative density",
                        title=self.title)
        self.ax.set_ylim(ymax=1)

    def pdf(self):
        self.n, self.bins, self.patches = self.ax.hist(x=self.data,
                        bins=self.bins,
                        align='left',
                        fill='true',
                        normed=1,
                        cumulative=False,
                        label=self.legend)
        self.setLabels( xlabel=self.xlabel,
                        ylabel="probability density",
                        title=self.title)

        cm = plt.cm.get_cmap('jet')
        for index, patch in enumerate(self.patches):
            plt.setp(patch, 'facecolor', cm(float(index/len(self.patches))))

    def boxplot(self):
        print(self.data)

        self.plot = plt.boxplot(self.data,
                                #notch=True,
                                patch_artist=True,
                                flierprops=dict(marker='x'))

        colors = ['steelblue', 'peachpuff', 'green', 'red', 'purple', 'brown', 'pink']
        color_repetitions = math.ceil(len(self.data)/len(colors))
        colors = color_repetitions * colors

        for patch, color in zip(self.plot['boxes'], colors):
            patch.set_facecolor(color)

        self.setLabels( xlabel="measurement",
                        ylabel=self.ylabel,
                        title=self.title)

        boxdict = self.ax.boxplot(self.data)
        fliers  = boxdict["fliers"]

        for j in range(len(fliers)):
            yfliers = boxdict['fliers'][j].get_ydata()
            xfliers = boxdict['fliers'][j].get_xdata()
            ufliers = set(yfliers)
            for i, uf in enumerate(ufliers):
                self.ax.text(xfliers[i] + 0.05, uf, list(yfliers).count(uf))

        if len(np.asarray(self.data).transpose()) == len(self.xticks):
            print(self.xticks)
            plt.xticks([x+1 for x in range(len(self.xticks))],self.xticks)
            #self.ax.set_xticklabels(self.xticks);
        else:
            print ( "len(self.data) = "
                    + str(len(self.data))
                    + " and len(self.xticks) = "
                    + str(len(self.xticks))
                    +" don't match!")

    def setLabels(self, xlabel="", ylabel="", title=""):
        self.ax.set_xlabel(xlabel)
        self.ax.set_ylabel(ylabel)
        self.ax.set_title(title)

    def save(self, savepath, plot_type):
        #savename = self.title
        savename = self.savename
        savename = savename.lower()
        savename = savename.replace(" ", "_")
        self.fig.savefig(savepath+savename+".png")
        if self.plot_pdf:
            self.fig.savefig(savepath+savename+".pdf")

    def show(self):
        plt.show()

    def annotate(self, annotations=False):
        if annotations:
			self.ax.annotate(annotations)
		# Add custom annotation code here
			...

    def debug(self):
        print("data: "+str(self.data))
        print("bins: "+str(self.bins))
        print("plottype: "+self.plottype)

\end{lstlisting}