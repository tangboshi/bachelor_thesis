\chapter{First Chapter Heading}

Use\footnote{This is a footnote to the word ``use''.}
\emph{The Not So Short Introduction to \LaTeXe}~\cite{Oetiker00}
to familiarize yourself with \LaTeX. When citing research papers, use a protected space ($\sim$) in front of the \texttt{cite} command, e.g. see this reference~\cite{inproceedings}.

For tables, see \url{http://www.inf.ethz.ch/personal/markusp/teaching/guides/guide-tables.pdf} on how to design them nicely. We are also giving an example in Table~\ref{tab:example}.

Learn how to properly manage your bibliography and ensure that if you cite~\cite{Achtzehn2015,ETSI2015,FCC,Fraga-Lamas2015} you use the right types and fields. Copying BibTeX blobs from the Internet is generally not a good idea.
\section{First Section Heading}

You can simply list a number of items as
\begin{itemize}
\item Bullet Item
\item Bullet Item
\end{itemize}

\noindent You can also enumerate them as
\begin{enumerate}
\item Bullet Item 
\item Bullet Item
\end{enumerate}
\subsection{First SubSection Heading}

Use the acronym package like this: \ac{SNR} % also \acs{}, acp{}, \acf{}, acl{}
You will need to specify each acronym in the abbreviations.tex file in the appendix. For more information, see \url{http://mirrors.ctan.org/macros/latex/contrib/acronym/acronym.pdf}.

\subsubsection{First SubSubSection Heading}

Referencing of figures: Use capital notation, e.g. ``In Figure~\ref{fig:inets-logo} we show the logo of the institute.''. Same applies for tables, e.g. ``Table~\ref{tab:example} lists the parameters we used in our simulation.'' However, do not capitalize in (non-numbered) case, e.g. ``It becomes apparent from the figure that\dots'' or ``The parameters listed in the table have been selected\dots''.
\begin{figure}\begin{center}
  \includegraphics[width=\textwidth]{pictures/inets_new} % width can be fractional, e.g. width=0.45\textwidth.
  \caption{iNETS logo.}\label{fig:inets-logo}
\end{center}\end{figure}

With numbering and alignment for all equations:
\begin{eqnarray}
\lambda&=&0.25 x^0 + 0.0001\\
       &=&2.5\cdot10^{-10} x^3 + 0.0001.
\end{eqnarray}


With partial numbering for equations:
\begin{eqnarray}
\lambda&=&0.25 x^0 + 0.0001 \nonumber \\
       &=&2.5\cdot10^{-10} x^3 + 0.0001.
\end{eqnarray}

\begin{table}
\centering
 \begin{center}
\begin{tabular}{p{2.8cm}p{0.2cm}p{1cm}rp{0.2cm}p{1cm}r}
\toprule
&&\multicolumn{2}{c}{set 1}&&\multicolumn{2}{c}{set 2}\\
\cmidrule{3-4}\cmidrule{6-7}
parameter &&value&precision&&value&precision\\
\midrule
param 1 [MHz]&& 1 & $\pm$1 && 2 & $\pm$1 \\
param 2 [s]&& 1 & $\pm$0.1 && 2 & $\pm$0.2 \\
\bottomrule
 \end{tabular}\caption{Table example.} \label{tab:example}
 \end{center}

\end{table}

\unit[25]{$\mu m$} = \unit[25\,000]{nm},
\unit[0]{$^\circ \text{C}$} = \unit[273.15]{K}, and
\unit[1]{Mbit/s} = \unit[$1\,024$]{kbit/s}. Always use SI prefixes (\url{http://physics.nist.gov/cuu/Units/prefixes.html}) in text, in comparative tables use engineering notation (\url{http://www.augustatech.edu/math/molik/notation.pdf})!

 When defining symbols, make sure to put multi-letter symbols/indexes into a proper text statement, e.g.
 \begin{eqnarray}
S_\text{off} \ge S_\text{on} \\
 \text{SINR} \le \text{SNR} 
 \end{eqnarray} 