\chapter{Conclusions and Future Work}

In the final chapter of the thesis, we will summarize the most important conclusions of the preceding Measurement Results chapter and give an outlook on what we think should be subject of future endeavors.

In Section \ref{sec:same-protocols} we showed that two nodes in symmetrical measurement setups, i.e. using the same MAC protocols with the same parameters in a shared channel perform similarly in terms of throughput, frame delay and backoff times where applicable. We showed that two backlogged pure ALOHA nodes recklessly push their packets into the channel resulting in zero throughput. Next, we showed that two CSMA/CA senders with DIFS, SIFS and BO values scaled up from the IEEE 802.11g standard coexist very well with almost collision-free and fair traffic. In an effort to increase throughput, we scaled the values of DIFS, SIFS and BO down and found out that we cannot arbitrarily reduce them without collisions\footnote{other than those resulting from the two nodes by chance transmitting at the same time}, due to the limited system time granularity caused by hardware delays. Solving the resulting optimization problem experimentally yielded $\text{DIFS=9ms}$, $\text{SIFS=1ms}$ and $\text{BO=2ms}$. In the last experiment of said section we proved that renouncing on the backoff mechanism leads to the typical 1-persistent CSMA behavior of multiple senders starting to transmit at the same time, leading to collision and thus high frame delays and high packet loss.

In Section \ref{sec:different-protocols} we examined the coexistence of transmitters employing different MAC protocols. Pro forma we have shown that a single saturated ALOHA sender, rather than two as in Section \ref{sec:same-protocols} is sufficient to lock out other senders from the channel. Consecutively, we showed that even low loads of ALOHA traffic are quite detrimental to the performance of another, in this case a CSMA/CA sender. The next experiment showed that in a scenario with two CSMA/CA senders the backoff slot duration\footnote{provided the minimum backoff window is the same and DIFS is rather small compared to the mean backoff or is scaled with backoff, which is the case in our experiments} is the decisive factor of which fraction of the aggregate throughput each sender gets. Subsequently, we tried to improve on the throughput of CSMA in the unsaturated ALOHA combined with CSMA/CA scenario by replacing the CSMA/CA sender with a 1-persistent CSMA sender, i.e. as a main measure remove the backoff. As a result, the 1-persistent CSMA throughput doubled, whereas the packet loss of ALOHA trippled to about 35\%\footnote{only we should mention, the reason is that SINR of the ALOHA sender was much higher than the SINR of the CSMA/CA sender}. The last experiment shows that from the POV of a CSMA/CA sender it does not make a difference whether it is a saturated 1-persistent CSMA or a saturated ALOHA sender occupying the shared channel.    

Eventually, we want to give an outlook on future work proceeding from questions undiscussed in this thesis. Going from our last experiment as a starting point it would interesting to see how much better 1-persistent CSMA fares compared to ALOHA in lower load situations concerning coexistence with CSMA/CA. More important and probably more practical questions are how the backoff mechanism can be enhanced to adaptively accommodate different traffic situations without introducing uneconomical complexity. One idea is to modify the CW growth by taking the CW of previous transmissions into account or to use a different growth scheme, e.g. linear instead of binary exponential. Another idea is using different backoff slot durations for different nodes in the network, particularly slot durations that are mutually prime to further reduce the chance that multiple nodes start sending at \emph{exact} same time. A completely different approach is to study the effect of transmission power, which may adaptively be varied to limit the range of a sender in order to allow a greater number of transmissions in the environment. On the other hand TX power could be adaptively increased if a node has priority traffic, allowing collisions to send traffic through a channel that is otherwise saturated. 
