\chapter{Conclusions and Future Work}
\label{ch:conclusions}

In this chapter, we summarize the most important conclusions from the measurement results and give an outlook on what we think should be subject of future endeavors.

We showed that two nodes in symmetrical measurement setups, i.e. using the same MAC protocols with the same parameters in a shared channel, perform similarly in terms of throughput, frame delay and backoff times, where applicable. We showed that two backlogged pure ALOHA nodes recklessly push their packets into the channel resulting in zero throughput. Next, we showed that two CSMA/CA transmitters with DIFS, SIFS and BO values scaled up from the IEEE 802.11g standard coexist very well with almost no collisions and similar throughput. In an effort to increase throughput, we scaled the values of DIFS, SIFS and BO down and found out that we cannot arbitrarily reduce them without collisions other than those resulting from the two nodes by chance transmitting at the same time, due to the limited system time granularity caused by hardware delays. Further optimizing the parameter values we found that $\text{DIFS}=9\,\text{ms}$, $\text{SIFS}=1\,\text{ms}$, $\text{BO}=2\,\text{ms}$ is still a stable combination. In the last experiment with the same MAC protocols for both links we proved that removing the backoff mechanism leads to the typical 1-persistent CSMA behavior of multiple transmitters starting to transmit at the same time, leading to collision and thus high frame delays and high packet loss.

Furthermore, we examined the coexistence of transmitters employing different MAC protocols. Pro forma we have shown that a single saturated ALOHA transmitter is sufficient to lock out other transmitters from the channel. Consecutively, we showed that even low loads of ALOHA traffic are quite detrimental to the performance of a CSMA/CA transmitter. The next experiment showed that in a scenario with two CSMA/CA transmitters the backoff slot duration, provided DIFS is rather small compared to the mean backoff, which is the case in our experiments, is the decisive factor for throughput distribution among links. Subsequently, we tried to improve on the throughput of CSMA in the unsaturated ALOHA combined with CSMA/CA scenario by replacing the CSMA/CA transmitter with a 1-persistent CSMA transmitter, i.e. as a main measure remove the backoff. As a result, the 1-persistent CSMA throughput doubled, whereas the packet loss of ALOHA trippled to about 35\%. The last experiment shows that from the point of view of a CSMA/CA transmitter it does not make a difference whether there is another saturated 1-persistent CSMA or saturated ALOHA transmitter occupying the shared channel.    

Eventually, we want to give an outlook on future work proceeding from questions left undiscussed in this thesis. Going from our last experiment as a starting point it would be interesting to see how much better 1-persistent CSMA fares compared to ALOHA in lower load situations concerning coexistence with CSMA/CA. More important and probably more practical questions are how the backoff mechanism can be enhanced to adaptively accommodate different traffic situations without introducing uneconomical complexity. One idea is to modify the CW growth by taking the average CW of a limited number of previous transmissions into account or to use a different growth scheme, e.g. linear instead of binary exponential. Other ideas are using different and/or adaptive backoff slot durations for different nodes in the network or changing minimum contention windows all in order to further reduce the chance that multiple nodes start sending at exact same time, while simultaneously keeping unnecessary waiting times at bay. 

Moreover, mechanisms of other MAC approaches, such as duty cycles, preamble sampling, piggybacking, as well as combinations of different mechanisms should be taken into consideration. Duty cycling as used in LTE-U or various MAC protocols for WSN seems particularly promising. 

A completely different approach is to study the effect of transmission power, which may adaptively be varied to limit the range of a transmitter in order to allow a greater number of transmissions in the environment. On the other hand TX power could be adaptively increased if a node has priority traffic, allowing collisions to send traffic through a channel that is otherwise saturated. 
