\chapter{Introduction}
\label{ch:introduction}

In this chapter we motivate the thesis and explain its structure by briefly summarizing the contents of each chapter.

\section{Motivation}
Licensees of dedicated frequency bands aim at extending their bandwidth by making use of unused capacities in unlicensed bands to accommodate the growing number of users and the demand for higher transfer rates. One example is LTE Unlicensed, which aggregates carriers in the license-free 5 GHz band already populated with dual-band WLAN devices. However, the original LTE technology was not designed to coexist with other technologies in the same channel. Particularly, LTE in the licensed bands relies on the fact that all access to the physical medium is coordinated by a base station. Another unlicensed band, namely the 2.4 GHz band, is currently much more populated. IEEE 802.11 (WLAN), IEEE 802.15.1 (Bluetooth) and IEEE 802.15.4 (ZigBee) devices all coexist in the 2.4 GHz band.

In order to facilitate harmonious coexistence of devices in the same channel the appropriate design of medium access control (MAC) protocols to avoid collisions is decisive, because collisions may render all transmitted data useless. The intention of this thesis is to examine how different MAC mechanisms and the choice of related parameters affect the performance in terms of throughput and other metrics. Our results are based on measurements with real, programmable devices making use of the flexibility of software-defined radio.   

\section{Structure of This Thesis}
In Chapter \ref{ch:background} we discuss the theoretical foundations of the ensuing experiments. We classify and introduce a number of MAC protocols considering strengths and weaknesses of some of their mechanisms. Furthermore, we will briefly introduce the main concepts of the software tool (GNU Radio) we used for our experiments. The purpose of Chapter \ref{ch:related-work} is to put our work into the context of related work, highlighting similarities and differences in the execution of the experiments and considered MAC aspects.
In Chapter \ref{ch:methodology} we give an overview of the conducted experiments. We provide all necessary information to reproduce our results. Moreover, we define the metrics that we have taken and how our automated measurement scripts greatly reduce the required user efforts to obtain results.
Chapter \ref{ch:results} contains the measurement results and our interpretation concerning the fitness of the protocols and their mechanisms for harmonious coexistence. 
In the final chapter we discuss the main findings of Chapter \ref{ch:results} and give an outlook on possible starting points for future work.