\chapter{Introduction}
\label{ch:introduction}

In this chapter we motivate the thesis and explain its structure by briefly summarizing the contents of each chapter.

A great number of technologies already operate in the unlicensed bands, however their number and density is still expected to increase. One example to this claim is that licensees of dedicated frequency bands aim at extending their bandwidth by making use of unused capacity in unlicensed bands to accommodate the growing number of users and the demand for higher transfer rates. This example is referring to LTE Unlicensed, which aggregates carriers in the license-free 5 GHz band already populated with Wi-Fi devices \cite{nihtilä13},\cite{qualcomm15}. However, the original LTE technology was not designed to coexist with other technologies in the same channel. Particularly, LTE in the licensed bands relies on the fact that all access to the physical medium is coordinated by a base station \cite{ghosh10}. Another unlicensed band, namely the 2.4 GHz band, is currently much more populated. IEEE 802.11 (Wi-Fi), Bluetooth and IEEE 802.15.4 (ZigBee) devices all coexist in the 2.4 GHz band \cite{lee07}. The specifications and standards of these three technologies already offer coexistence mechanisms especially in view of rapid network densification \cite{bhushan14}. In order to facilitate harmonious coexistence of devices in the same channel the appropriate design of medium access control (MAC) protocols to avoid collisions is decisive, because collisions may render all transmitted data useless. The goal of this thesis is to examine how different MAC mechanisms and the choice of related parameters affect the network performance in terms of throughput and other metrics. Our results are based on measurements with universal software radio peripherals (USRPs) which are physical, programmable devices, making use of the flexibility of software-defined radios. In contrast to other inter-technology coexistence studies, such as \cite{gomezmiguelez16} and \cite{capretti16} on LTE-U/Wi-Fi coexistence, \cite{zhang11} on Zigbee/Wi-Fi coexistence and \cite{corvaja06},\cite{chiasserini02} on Bluetooth/Wi-Fi coexistence, we focus on timing aspects of CSMA/CA, the MAC protocol used in Wi-Fi, such as interframe spacing, backoff slot duration and contention window.

The rest of the thesis is structured as follows. In Chapter \ref{ch:background} we discuss the theoretical foundations for the ensuing experiments. We classify and introduce a number of MAC protocols considering their strengths and weaknesses. Furthermore, we will briefly introduce the main concepts of the software tool GNU Radio, that we used for our experiments. The purpose of Chapter \ref{ch:related-work} is to put our work into the context of related work.
In Chapter \ref{ch:methodology} we give an overview of the conducted experiments. Moreover, we define the metrics that we have considered and show how our automated measurement scripts greatly reduce the required user effort to obtain results.
Chapter \ref{ch:results} contains the measurement results and our interpretation concerning the fitness of the protocols for harmonious coexistence. 
In chapter \ref{ch:conclusions} we discuss the main findings of Chapter \ref{ch:results} and give an outlook on possible starting points for future work.