\chapter{Related Work}
\label{ch:related-work}

This chapter introduces work related to the thesis. We will highlight similarities and differences of their approaches to our work. 

\section{Inter-Technology Coexistence}

In the following section we discuss the results of some studies that examined the coexistence of different technologies in the same frequency band. Studies concerning inter-technology coexistence can be split into two groups. The first group of studies is based on simulations, whereas the second is based on measurements with physical devices. Due to the fact, that we also carry out experimental research using USRPs, we are more interested in the latter group. Furthermore, studies based on real devices have the benefit of better reflecting system level details of the technologies and providing insight for real-world deployments. However, as pointed out in \cite{gomezmiguelez16} vendor-specific effects of the test hardware must be taken into account since they may exert great influence on the results. 

\paragraph{LTE-U/Wi-Fi Coexistence}
Although a number of simulation-based studies \cite{nihtilä13}, \cite{rupasinghe14}, \cite{jeon14} and \cite{cavalcante13} exist on this topic we will confine the discussion to two studies \cite{gomezmiguelez16}, \cite{capretti16} based on measurements with physical devices. Both studies evaluate LTE Unlicensed /Wi-Fi coexistence based on LTE-U using srsLTE, an open-source SDR library to implement the PHY layer of LTE release 8. Another common feature of both studies is the use of commodity Wi-Fi network interface cards and USRPs as LTE nodes. 

In \cite{gomezmiguelez16} the testbed comprises of several Wi-Fi and LTE links, for which they used Ettus USRP B210 boards (LTE) and low-power single-board computers from Soekris (Wi-Fi). In order to detect vendor-specific performance issues they decided to use two different sets of wireless NICs from Atheros and Broadcomm. In their study the influence of the following parameters was examined: LTE-U duty cycle, Wi-Fi and LTE TX power, LTE bandwidth, LTE central frequency (i.e. LTE and Wi-Fi spectrum overlap). Their main results can be summarized as follows:
\begin{itemize}
	\item  Wi-Fi throughput is inversely proportional to LTE duty cycle.
	\item  Wi-Fi TX power has little impact on Wi-Fi throughput.
	\item  The influence of LTE bandwidth and central frequency on Wi-Fi throughput depends very much on the vendor of the NIC card. As a consequence, more experimental research with physical devices from different vendors is strongly recommended. 
\end{itemize}
  
The testbed in\cite{capretti16} consists of one LTE base station (eNodeB or eNB) and one user equipment (UE), one Wi-Fi access point and five other Wi-Fi nodes. Their Wi-Fi network was based on embedded PCs equipped with commodity wireless adapters. The LTE nodes were based on desktop computers with Ettus USRP B210 RF front ends running the open-source driver UHD. An interesting detail is that they also used GNU Radio. The following parameters were subject of interest: duty cycle, Wi-Fi power settings Wi-Fi MCS (modulation and coding scheme) and packet size. The metrics measured were satisfied load in percent, total Wi-Fi throughput, Wi-Fi jitter and LTE packet loss.  Their main findings can be summarized as follows: 
\begin{itemize}
	\item The duty cycle patterns are a main influence on achievable Wi-Fi throughput. Particularly, shorter duty cycles decrease jitter, which is important for real-time applications. On the other hand longer duty cycles offer superior throughput due to reduced overhead.
	\item LTE suppresses Wi-Fi transmissions if the TX power levels are comparable and no duty cycling is employed.
	\item If Wi-Fi TX power is increased, Wi-Fi load negatively impacts LTE throughput. There is no panacea strategy ensuring maximum Wi-Fi throughput operating under different MCSs and packet sizes. LTE performance is unaffected by Wi-Fi contention levels.
\end{itemize}

\paragraph{ZigBee/Wi-Fi Coexistence}


\paragraph{Bluetooth/Wi-Fi Coexistence}

\section{MAC Protocol Modeling}

In this section we discuss approaches to modeling MAC protocols.

\paragraph{Finite State Machines}

\paragraph{Data Flow}


