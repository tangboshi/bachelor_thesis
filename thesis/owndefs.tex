\usepackage[latin1]{inputenc} % latin1 for Windows and older Linuxes,
                              % utf8 for newer Linuxes
\usepackage[T1]{fontenc}

\usepackage{physics}
\usepackage{svg}
\usepackage[export]{adjustbox}
\usepackage{listings}
\usepackage{color}

\definecolor{dkgreen}{rgb}{0,0.6,0}
\definecolor{gray}{rgb}{0.5,0.5,0.5}
\definecolor{mauve}{rgb}{0.58,0,0.82}

\lstset{frame=left,
	aboveskip=7mm,
	belowskip=7mm,
	captionpos=b,
	showstringspaces=false,
	columns=flexible,
	basicstyle={\small\ttfamily},
	numbers=left,
	numbersep=7pt,
	keywordstyle=\color{mauve},
	commentstyle=\color{gray},
	stringstyle=\color{dkgreen},
	numberstyle=\color{gray},
	breaklines=true,
	breakatwhitespace=true,
	tabsize=4,
	keepspaces=true
}


%\usepackage{marvosym} % Martin Vogel's symbol package
%\usepackage{rotating} % Rotate things as you like
%\usepackage{lscape}   % Rotate text to landscape
%\usepackage{fancyvrb} % Fancy package for reading and writing verbatim TeX code

% ... and environments
\renewenvironment{quote}%
   {\begin{quotation}\noindent\itshape}%
   {\upshape\end{quotation}}

% ----- Hyphenation! -----
% Now special cases could be defined and how these should always be hyphenated.
\hyphenation{ }
